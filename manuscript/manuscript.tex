% !TeX root = RJwrapper.tex
\title{\texttt{quickmapr}: Simplified mapping and basic interactivy for
\texttt{sp} and \texttt{raster} objects.}
\author{by Jeffrey W. Hollister}

\maketitle

\abstract{%
There are many packages that already exist or are in active development
that support the visualization of spatial data in R. However, there
seems to be a gap for those that need to quickly view, compare, and
interactively explore the results of a given spatial analysis without
first having to convert a different coordinate reference system.
Functionality for the current release (v0.2.0) is for easy mapping of
multiple layers, simple zooming, panning, labelling, and identifying.
These tools are intended for use within an active spatial analysis
workflow and not for production quality maps. Additionally,
\texttt{quickmapr} does not make any assumptions about coordinate
reference systems and leaves managing of projections to the analyst.
This paper introduces the package and shows examples of its typical use.
}

\subsection{Introduction}\label{introduction}

Great sptatial data stuff in R

Visualization tools too, but no easy way to interact with the mapped
data.

Recently, lots of effort of on spatial data viz (e.g.~ggmaps, leaflet,
cartographer etc.) that rely on javascript libraries or other web APIs.
These provide a modern interface, with a rich set of basemaps, but all
assume and unprojected or Web Mercator coordinate reference system. In
the case of typical spatial data analysis workflow it is often desirable
to quickly map the resultant spatial datasets in the projection chosen
for the analysis. Currently, this is not possible with the most used
javascript libraires. I developed \texttt{quickmapr} to fill this gap
and provide spatial data analysts with a tool to quickly map multiple
layers and interact with the resultant map without having to utilize
various APIs or external libraries and without having to re-project
data.

This paper describes the basic usage of \texttt{quickmapr} and shows
examples\ldots{}

\subsection{\texorpdfstring{Basic usage of
\texttt{quickamapr}.}{Basic usage of quickamapr.}}\label{basic-usage-of-quickamapr.}

This section may contain a figure such as Figure \ref{figure:rlogo}.

\begin{figure}[htbp]
  \centering
  \includegraphics{Rlogo}
  \caption{The logo of R.}
  \label{figure:rlogo}
\end{figure}

\subsection{\texorpdfstring{The \texttt{qmap} function and
object}{The qmap function and object}}\label{the-qmap-function-and-object}

There will likely be several sections, perhaps including code snippets,
such as:

\begin{Schunk}
\begin{Sinput}
x <- 1:10
x
\end{Sinput}
\begin{Soutput}
#>  [1]  1  2  3  4  5  6  7  8  9 10
\end{Soutput}
\end{Schunk}

\subsection{Zooming and panning}\label{zooming-and-panning}

\subsection{Identification and
selection}\label{identification-and-selection}

\subsection{Basemaps from the USGS National
Map}\label{basemaps-from-the-usgs-national-map}

\subsection{Summary}\label{summary}

This file is only a basic article template. For full details of
\emph{The R Journal} style and information on how to prepare your
article for submission, see the
\href{https://journal.r-project.org/share/author-guide.pdf}{Instructions
for Authors}.

\bibliography{RJreferences}

\address{%
Jeffrey W. Hollister\\
U.S. Environmental Protection Agency\\
Office of Research and Development\\ National Health and Environmental Effects Research Laboratory\\ Atlantic Ecology Division\\ 27 Tarzwell Drive\\ Narragansett, RI 02882\\
}
\href{mailto:hollister.jeff@epa.gov}{\nolinkurl{hollister.jeff@epa.gov}}

