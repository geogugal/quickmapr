% !TeX root = RJwrapper.tex
\title{quickmapr: Simplified mapping and basic interactivy for \texttt{sp} and
\texttt{raster} objects.}
\author{by Jeffrey W. Hollister}

\maketitle

\abstract{%
An abstract of less than 150 words.
}

\subsection{Introduction}\label{introduction}

Introductory section which may include references in parentheses
\citep{R}, or cite a reference such as \citet{R} in the text.

\subsection{Section title in sentence
case}\label{section-title-in-sentence-case}

This section may contain a figure such as Figure \ref{figure:rlogo}.

\begin{figure}[htbp]
  \centering
  \includegraphics{Rlogo}
  \caption{The logo of R.}
  \label{figure:rlogo}
\end{figure}

\subsection{Another section}\label{another-section}

There will likely be several sections, perhaps including code snippets,
such as:

\begin{Schunk}
\begin{Sinput}
x <- 1:10
x
\end{Sinput}
\begin{Soutput}
#>  [1]  1  2  3  4  5  6  7  8  9 10
\end{Soutput}
\end{Schunk}

\subsection{Summary}\label{summary}

This file is only a basic article template. For full details of
\emph{The R Journal} style and information on how to prepare your
article for submission, see the
\href{https://journal.r-project.org/share/author-guide.pdf}{Instructions
for Authors}. \bibliography{RJreferences}

\address{%
Jeffrey W. Hollister\\
U.S. Environmental Protection Agency, Office of Research and
Development, National Health and Environmental Effects Research
Laboratory, Atlantic Ecology Division\\
27 Tarzwell Drive\\ Narragansett, RI 02882\\
}
\href{mailto:hollister.jeff@epa.gov}{\nolinkurl{hollister.jeff@epa.gov}}

